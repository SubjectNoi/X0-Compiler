\documentclass{article}
	\usepackage{booktabs}
	\usepackage{listings}
	\usepackage{geometry}
	\geometry{left=2cm, right=2cm, top=2cm, bottom=2cm}
	\usepackage{xcolor}
	\usepackage{amsmath}
	\title{X0-Compiler Design Document}
	\usepackage[colorlinks, linkcolor=red]{hyperref}
	\author{Altair, Liu @ ilovehanhan1120@hotmail.com}
	\begin{document}
		\maketitle
		\section{Introduction}
		\subsection{Purpose}
		The purpose of conducting as technical proposal to describe the global designing of this project, containing basic functionality of the system, run-time designing and error detecting methods. This document is aimed to provide a schema of designing and implement all functionality, which will be the critical document during the process of developing. This document will be read by developers and testers.
		\subsection{Background}
		This project is to develop a \textbf{X0 Language Compiler}, which is a C-like language. This project is mainly for research and study purpose.
		\begin{center}
		\begin{tabular}{cc}
			\toprule
			Item & Detail\\
			\midrule
			Project Name & X0-Compiler(mini-C)\\
			Developing Platform & Ubuntu 18.04 64-bit\\
			Developing Tools & \textbf{Flex} and \textbf{Bison}\\
			Open Source or not & Yes \\
			\bottomrule 
		\end{tabular}\\
		\end{center}
		All source files can be found at: \url{http://github.com/SubjectNoi/X0-Compiler}
		\subsection{Remarks}
		Usage:
		\begin{lstlisting}[language={sh},numbers=left,numberstyle=\tiny,%frame=shadowbox,  
   rulesepcolor=\color{red!20!green!20!blue!20},  
   keywordstyle=\color{blue!70!black},  
   commentstyle=\color{blue!90!},  
   basicstyle=\ttfamily]  
 Ubuntu>$ git clone http://github.com/SubjectNoi/X0-Compiler
 Ubuntu>$ cd X0-Compiler
 Ubuntu>$ make
 Ubuntu>$ ./X0 [Your source file]
\end{lstlisting}  
		\section{Design Summarize}
		\subsection{Main purpose of the project}
		Following are main purposes of this project:
		\begin{itemize}
		\item Run correctly on target OS: Ubuntu 18.04 64-bit
		\item Compile X0 language
		\item Report compile error, including syntax and semantic error 
		\end{itemize}
		\subsection{Primary demand}
		The X0 compiler should compile these C-like language, detailed grammar definition will be showed in next section.
		\begin{lstlisting}[language={C},numbers=left,numberstyle=\tiny,%frame=shadowbox,  
   rulesepcolor=\color{red!20!green!20!blue!20},  
   keywordstyle=\color{blue!70!black},  
   commentstyle=\color{blue!90!},  
   basicstyle=\ttfamily]  
main {
	integer i, j, flag, cnt := 0;
	for (i := 2; i != 101; i++) {
		flag := 0;
		for (j := 2; j != i; j++) { 
			if (i % j == 0) {
				flag := 1;
				break;
			}
		}
		if (flag == 0) {
			write(i);
			cnt++;
		}
	}
	write("There're:");
	write(cnt);
	write("Primes.");
}
\end{lstlisting}
		And correct result should be given. If there exists syntax or semantic error, compiler should report them.
		\subsection{Restrictions of Design}
		To complete this project, following restrictions should be watched out:
		\begin{itemize}
		\item Project will be only run on Ubuntu 18.04
		\item Both developing and testing should be finished before 2018-11-26T11:30:00.000Z
		\end{itemize}
		\subsection{Principles and Rules of Design}
		Following principles should be followed in the process of developing:
		\begin{itemize}
		\item Complete: implement as many features as possible
		\item Simple: try best to ensure low coupling between modules
		\item High Efficiency: try best to ensure the highest execution efficiency of virtual machine code.
		\end{itemize}
		When developing, following rules should be obey:
		\begin{itemize}
		\item All files should be named under following rules:
		\begin{center}
		\begin{tabular}{cc}
			\toprule
			File & Naming rule\\
			\midrule
			Yacc file & X0-Bison.y\\
			Lex file & X0-Lex.l\\
			Constructing file & Makefile\\
			Testing source & /TestingSrc/Test\textbf{XX}\_\textbf{[Testing Content]} \\
			Git ignore file & .gitignore \\
			\bottomrule 
		\end{tabular}\\
		\end{center}
		\item Git is used for version control
		\item Use \textbf{git fetch \&\& git pull}
		\item Use \textbf{git rm -r --cached .}
		\item Use \textbf{git add .}
		\item Use \textbf{git commit -am [Meaningful Comment]}
		\item Never \textbf{git push -f}
		\end{itemize}
		\section{Main Design}
		\subsection{Demand}
		In this sub-section, detailed grammar of X0 Language will be given:
		\begin{align}
		\textbf{program} \rightarrow & 'main', \lbrace, \textbf{statement\_list}, \rbrace \\
		\textbf{statement\_list} \rightarrow &
		\textbf{statement\_list}, \textbf{statement} \\ 
	  &| \textbf{statment} \\ 
	  &| \epsilon \\
		\textbf{statement} \rightarrow &
		\textbf{expression\_list} \\ 
	  &| \textbf{if\_statement} \\ 
	  &| \textbf{while\_statement} \\
	  &| \textbf{read\_statement} \\ 
	  &| \textbf{switch\_statement} \\ 
	  &| \textbf{case\_stat} \\ 
	  &| \textbf{write\_statement} \\
	  &| \textbf{compound\_statement} \\ 
	  &| \textbf{for\_statement} \\
	  &| \textbf{do\_statement} \\ 
	  &| \textbf{declaration\_list} \\
	  &| \textbf{continue\_stat} \\ 
	  &| \textbf{break\_stat} \\
	  &| \textbf{yarimasu\_stat}\\
	  &| \epsilon \\
		\textbf{declaration\_list} \rightarrow &
		\textbf{declaration\_list}, \textbf{declaration\_stat}\\ 
	  &| \textbf{declaration\_stat} \\ 
	  &| \epsilon \\
	  	\textbf{declaration\_stat} \rightarrow & 
	  	\textbf{typeenum},\textbf{identlist},';'\\
	  &|\textbf{typeenum},\textbf{identarraylist}\\
	  &|'const',\textbf{typeenum},\textbf{identlist},\textbf{SEMICOLONSTAT}\\
	  &|'const',\textbf{typeenum},\textbf{identarraylist}\\
	  \textbf{identlist} \rightarrow & 
	  \textbf{identdef} \\
	  &|\textbf{identlist}, ',', \textbf{identdef} \\
	  &|\epsilon \\
	  \textbf{identdef} \rightarrow &
	  IDENT \\
	  &|IDENT, ':=', \textbf{factor} \\
	  \end{align}
	  \begin{align}
	  \textbf{typeenum} \rightarrow &
	  'integer' \\
	  &|'string' \\
	  &|'bool'	\\
	  &|'real'\\
	  &|'char'\\
	  \textbf{identarraylist} \rightarrow &
	  \textbf{identarraydef} \\
	  &|\textbf{identarraylist},',',\textbf{identarraydef}\\
	  \textbf{identarraydef} \rightarrow &
	  IDENT, '[', \textbf{dimensionlist}, ']'\\
	  \textbf{dimensionlist} \rightarrow &
	  \textbf{dimension} \\
	  &|\textbf{dimensionlist},',',\textbf{dimension}\\
	  \textbf{dimension} \rightarrow &
	  INTEGER \\
	  \textbf{switch\_statement} \rightarrow &
	  'switch','(',\textbf{expression},')', '\lbrace', \textbf{case\_list},\textbf{default\_statement}, '\rbrace' \\
	  \textbf{case\_list} \rightarrow &
	  \textbf{case\_list},\textbf{case\_stat} \\
	  &|\textbf{case\_stat} \\
	  &|\epsilon\\
	  \textbf{case\_stat} \rightarrow &
	  'case',\textbf{expression},':',\textbf{compound\_statement}\\
	  &|\epsilon\\
	  \textbf{default\_statement} \rightarrow &
	  'default',':',\textbf{compound\_statement}\\
	  \textbf{continue\_stat} \rightarrow & 
	  'continue',';'\\
	  \textbf{break\_stat} \rightarrow & 
	  'break',';'\\
	  \textbf{if\_statement} \rightarrow &
	  'if','(',\textbf{expression},')',\textbf{compound\_statement},\textbf{else\_list}\\
	  \textbf{else\_list} \rightarrow &
	  'else',\textbf{compound\_statement}\\
	  &|\epsilon\\
	  \textbf{while\_statement} \rightarrow &
	  'while','(',\textbf{expression},')',\textbf{compound\_statement}\\
	  \textbf{write\_statement} \rightarrow &
	  'write','(',\textbf{expression},')'\\
	  \textbf{read\_statement} \rightarrow &
	  'read','(',\textbf{var},')'\\
	  \textbf{compound\_statement} \rightarrow &
	  '\lbrace', \textbf{statement\_list}, '\rbrace'\\
	  \textbf{for\_statement} \rightarrow &
	  'for','(',\textbf{expression},';',\textbf{expression},';',\textbf{expression},')',\\ &\textbf{compound\_statement}\\
	  \textbf{do\_statement} \rightarrow &
	  'do',\textbf{compound\_statement},'while','(',\textbf{expression},')',';'\\
	  \textbf{var} \rightarrow &
	  IDENT\\
	  &|IDENT, '[',\textbf{expression\_list},']'\\
	  \textbf{expression\_list} \rightarrow &
	  \textbf{expression} \\
	  &|\textbf{expression\_list}, ',', \textbf{expression}\\
	  \textbf{expression} \rightarrow &
	  \textbf{var},':=',\textbf{expression} \\
	  &|\textbf{simple\_expr}\\
	  \textbf{simple\_expr} \rightarrow &
	  \textbf{additive\_expr} \\
	  &|\textbf{additive\_expr},\textbf{OPR},\textbf{additive\_expr}\\
	  &|\textbf{additive\_expr},\textbf{SINGLEOPR}\\
	  &|\textbf{SINGLEOPR},\textbf{additive\_expr}\\
	  \textbf{SINGLEOPR} \rightarrow &
	  '++' | '--' | '!'\\
	  \textbf{OPR} \rightarrow & 
	  '=='|'!='|'<'|'<='|'>'|'>='|'\&\&'|'||'|'\wedge\wedge'|'<<'|'>>'\\
		\end{align}
		\begin{align}
		\textbf{additive\_expr} \rightarrow &
		\textbf{term} \\
		&|\textbf{additive\_expr}, \textbf{PLUSMINUS},\textbf{term}\\
		\textbf{PLUSMINUS} \rightarrow &
		'+'|'-'\\
		\textbf{term} \rightarrow &
		\textbf{factor} \\
		&| \textbf{term},\textbf{TIMESDIVIDE},\textbf{factor}\\
		\textbf{TIMESDEVIDE} \rightarrow &
		'*'|'/'|'\%'\\
		\textbf{factor} \rightarrow &
		'(',\textbf{expression},')'\\
		&|\textbf{var}\\
		&|INTEGER \\
		&|REAL \\
		&|STRING \\
		&|BOOL\\
		&|CHAR\\
		&|YAJU\\
		\textbf{yarimasu\_stat} \rightarrow &
		'yarimasune',';'\\
		\end{align}
		This language should follow this grammar, detailed development of every modules will be mentioned below.
		\subsection{Environment}
		This project is developed on Ubuntu 18.08 64-bit, using \textbf{make} and corresponding \textbf{Makefile} to construct. External tool needed are: \textbf{Bison}, \textbf{Flex}, \textbf{VsCode}, \textbf{git}.
		\subsection{Modules}
		\subsection{Description of Modules}
		\subsection{Hardware}
		
		\section{Details of modules developing}
		
	\end{document}